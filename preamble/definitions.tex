%% Lengths

% base unit for coordinates
\newlength\quantum
\newlength\quanta
\setlength\quantum{3pt}
\setlength\quanta{\quantum}

% slides dimensions = N * quantum,
% ratio 120/90 = 4/3, with 3 quanta
% for the footline
\setlength{\paperwidth}{120\quanta}
\setlength{\paperheight}{93\quanta}

% set a footline with total height (height + depth) of 3 quanta
\newlength\tinyxheight
\newlength\flheight
\newlength\fldepth
\settoheight{\tinyxheight}{\tiny x}
\setlength\flheight{\dimexpr 1.5\quantum + 0.5\tinyxheight \relax}
\setlength\fldepth{\dimexpr 1.5\quantum - 0.5\tinyxheight \relax}

% width of a column in a five (V) columns layout
\newlength\colv
\newlength\colvsep  % including the inter-column spacing
\setlength\colv{17.6\quanta}
\setlength\colvsep{\dimexpr\colv + \baselineskip\relax}

% big column width (two columns + one baseline skip)
\newlength\bigcol
\newlength\halfbigcol
\setlength\bigcol{\dimexpr 2\colv + \baselineskip\relax}
\setlength\halfbigcol{0.5\bigcol}

% depth in Large fonts, for positioning titles
\newlength\Largeydepth
\settodepth{\Largeydepth}{\Large y}

% M height in footnotesize fonts, for positioning tick labels
\newlength\ssMheight
\settoheight{\ssMheight}{\scriptsize M}

% temporary lengths to locally replace long length calculations
\newlength\templen
\newlength\buffer

%% lists stuff
\newenvironment{listed}{
	\begin{itemize}[nosep]
	\setlength\itemsep{2\quanta}  % adds to the existing baseline skip
	}{
	\end{itemize}
	}

% remove indentation
\settowidth{\leftmargini}{\usebeamertemplate{itemize item}}
\addtolength{\leftmargini}{.36\labelsep} % narrower than usual item symbol

%% Overall style

\colorlet{col}{orange}  % main color

% square bullet items
\setbeamertemplate{itemize item}[square]
\setbeamercolor{itemize item}{fg=col}

% enumitem overrides beamer settings,
% so list styles have to be re-specified
\setitemize{label=\usebeamerfont*{itemize item}%
  \usebeamercolor[fg]{itemize item}
  \usebeamertemplate{itemize item}}



%% TikZ based slide template

% slide environment composed of a TikZ picture
% with the following nodes
%
% (NW) ----- (N) ----- (NE)
%   |         |         |
%   |         |         |
%  (W) ----- (C) ----- (E)
%   |         |         |
%   |         |         |
% (SW) ----- (S) ----- (SE)

\NewEnviron{slide}[1][]{%
\begin{frame}[t]%
	\makebox[\textwidth][c]{%
		\begin{tikzpicture}[overlay, remember picture, inner sep=0pt]%
		
		\node (N) at (page cs:0,45) {};
		\node (S) at (page cs:0,-45) {};
		\node (W) at (page cs:-60,0) {};
		\node (E) at (page cs:60,0) {};
		\node (NW) at (page cs:-60,45) {};
		\node (NE) at (page cs:60,45) {};
		\node (SW) at (page cs:-60,-45) {};
		\node (SE) at (page cs:60,-45) {};
		\node (C) at (page cs:0,0) {};
		
		\node[anchor=north west, align=left, font=\Large, black!60!gray,
			  yshift=\Largeydepth+.95pt, xshift=-.95pt] (title)
			at (page cs:-52,37) {#1};
			
			\normalsize\BODY
		
		\end{tikzpicture}}%
\end{frame}}

% draw help lines
\newcommand\drawhelplines{
	% title top lines
	\draw[teal] (page cs:-60,37) -- (page cs:60,37)
				(page cs:-60,31) -- (page cs:60,31);
	% pictures grid
	\draw[gray] foreach \c in {0,...,4}{
				(p5cl cs:\c,0)  -- (p5cl cs:\c,15)
				(p5cr cs:\c,0)  -- (p5cr cs:\c,15)
					foreach \l in {0,...,15}{
						(p5cl cs:\c,\l)  -- (p5cr cs:\c,\l)}};
	% text grid
	\draw[teal] foreach \c in {0,...,4}{
					foreach \l in {0,...,14}{
						(t5cl cs:\c,\l)  -- (t5cr cs:\c,\l)}};
	% margin
	\draw[col] (p5cr cs:1,-2) node (x) {} -- (x |- N);}

% draw full grid
\newcommand\drawfullgrid{
	\draw[gray] foreach \x in {-60,...,60}{
					(page cs:\x,-45) -- (page cs:\x,45)}
				foreach \y in {-45,...,45}{
					(page cs:-60,\y) -- (page cs:60,\y)};}



%% Plot options
\pgfkeys{/pgfplots/plot options/.style={
    separate axis lines,
    enlargelimits=false,
    axis x line*=bottom,
    axis x line shift=4\quanta,
    axis y line shift=4\quanta,
    axis y line*=left,
    scale only axis,
    ticklabel style={color=gray},
    axis line style={gray!80!white, semithick},
    major tick style={gray!80!white, semithick},
    ticklabel style={font=\scriptsize},
    xticklabel style={yshift=\ssMheight-3\quanta},
    yticklabel style={xshift=\ssMheight-3\quanta},}}



%% Footline

% white on black footline, to set it apart from the slide
\setbeamercolor{head/foot}{fg=white, bg=black}
\setbeamertemplate{footline}{
\begin{beamercolorbox}[wd=\paperwidth,
					   ht=\flheight,
					   dp=\fldepth]{head/foot}
	\makebox[.3333\paperwidth]{\insertsection}
	\makebox[.3333\paperwidth]{\itshape\insertsubsection}
	\hfill\insertframenumber{}\hspace*{2\quantum}
\end{beamercolorbox}%
}
\beamertemplatenavigationsymbolsempty  % remove navigation symbols


%% Quantities
\newcommand{\xmath}[1]{\ensuremath{#1}\xspace}  % ensure math mode

\newcommand{\Qh}[1][]{\xmath{\skew{5}\dot Q_{\text{h}#1}}}
\newcommand{\Qc}{\xmath{\skew{5}\dot Q_\text{c}}}
\newcommand{\Qcs}{\xmath{\skew{5}\dot Q_\text{cs}}}
\newcommand{\Qcl}{\xmath{\skew{5}\dot Q_\text{cl}}}
\newcommand{\Pel}{\xmath{P_\text{el}}}
\newcommand{\Pcp}{\xmath{P_\text{comp}}}
\newcommand{\Pfi}{\xmath{P_\text{fi}}}
\newcommand{\fc}[1][]{\xmath{f_{c#1}}}
\newcommand{\fcn}[1][]{\xmath{\nu_{#1}}}
\newcommand{\Tr}[1][]{\xmath{T_{r#1}}}
\newcommand{\Twbr}[1][]{\xmath{T_{{\text{wb}r#1}}}}
\newcommand{\Ts}{\xmath{T_s}}
\newcommand{\To}[1][]{\xmath{T_{o#1}}}
\newcommand{\Tset}{\xmath{T_\text{set}}}
\newcommand{\dTr}{\xmath{\Delta T_r}}
\newcommand{\mr}{\xmath{\skew{5}\dot m_\text{r}}}
\newcommand{\ma}[1][]{\xmath{\skew{5}\dot m_{\text{a}#1}}}
\newcommand{\tdf}{\xmath{\tau_\text{df}}}
\newcommand{\trec}{\xmath{\tau_\text{rec}}}
\newcommand{\tss}{\xmath{\tau_\text{ss}}}
\newcommand{\theat}{\xmath{\tau_\text{h}}}
\newcommand{\Va}[1][]{\xmath{\dot V_{a#1}}}
