%% Lengths

% base unit for coordinates
\newlength\quantum
\newlength\quanta
\setlength\quantum{3pt}
\setlength\quanta{\quantum}

% slides dimensions = N * quantum,
% ratio 120/90 = 4/3, with 3 quanta
% for the footline
\setlength{\paperwidth}{120\quanta}
\setlength{\paperheight}{93\quanta}

% set a footline with total height (height + depth) of 3 quanta
\newlength\tinyxheight
\newlength\flheight
\newlength\fldepth
\settoheight{\tinyxheight}{\tiny x}
\setlength\flheight{\dimexpr 1.5\quantum + 0.5\tinyxheight \relax}
\setlength\fldepth{\dimexpr 1.5\quantum - 0.5\tinyxheight \relax}

% width of a column in a five (V) columns layout
\newlength\colv
\newlength\colvsep  % including the inter-column spacing
\setlength\colv{17.6\quanta}
\setlength\colvsep{\dimexpr\colv + \baselineskip\relax}

% big column width (two columns + one baseline skip)
\newlength\bigcol
\newlength\halfbigcol
\setlength\bigcol{\dimexpr 2\colv + \baselineskip\relax}
\setlength\halfbigcol{0.5\bigcol}

% depth in Large fonts, for positioning titles
\newlength\Largeydepth
\settodepth{\Largeydepth}{\Large y}

% M height in footnotesize fonts, for positioning tick labels
\newlength\ssMheight
\settoheight{\ssMheight}{\scriptsize M}

% temporary lengths to locally replace long length calculations
\newlength\templen
\newlength\buffer

% map layout
\newlength\w
\newlength\e

%% lists stuff
\newenvironment{listed}{
	\begin{itemize}[nosep]
	\setlength\itemsep{2\quanta}  % adds to the existing baseline skip
	}{
	\end{itemize}
	}

% remove indentation
\settowidth{\leftmargini}{\usebeamertemplate{itemize item}}
\addtolength{\leftmargini}{.36\labelsep} % narrower than usual item symbol

%% Overall style

\colorlet{col}{orange}  % main color
% map colors
\colorlet{hls1}{black}
\colorlet{hls2}{black}
\colorlet{hls3}{black}

% square bullet items
\setbeamertemplate{itemize item}[square]
\setbeamercolor{itemize item}{fg=col}

% enumitem overrides beamer settings,
% so list styles have to be re-specified
\setitemize{label=\usebeamerfont*{itemize item}%
  \usebeamercolor[fg]{itemize item}
  \usebeamertemplate{itemize item}}



%% TikZ based slide template

% slide environment composed of a TikZ picture
% with the following nodes
%
% (NW) ----- (N) ----- (NE)
%   |         |         |
%   |         |         |
%  (W) ----- (C) ----- (E)
%   |         |         |
%   |         |         |
% (SW) ----- (S) ----- (SE)

\NewEnviron{slide}[1][]{%
\begin{frame}[t]%
	\makebox[\textwidth][c]{%
		\begin{tikzpicture}[overlay, remember picture, inner sep=0pt]%
		
		\node (N) at (page cs:0,45) {};
		\node (S) at (page cs:0,-45) {};
		\node (W) at (page cs:-60,0) {};
		\node (E) at (page cs:60,0) {};
		\node (NW) at (page cs:-60,45) {};
		\node (NE) at (page cs:60,45) {};
		\node (SW) at (page cs:-60,-45) {};
		\node (SE) at (page cs:60,-45) {};
		\node (C) at (page cs:0,0) {};
		
		\node[anchor=north west, align=left, font=\Large, black!80,
			  yshift=\Largeydepth+.95pt, xshift=-.95pt] (title)
			at (page cs:-52,37) {#1};
			
			\normalsize\BODY
		
		\end{tikzpicture}}%
\end{frame}}

% draw help lines
\newcommand\drawhelplines{
	% title top lines
	\draw[teal] (page cs:-60,37) -- (page cs:60,37)
				(page cs:-60,31) -- (page cs:60,31);
	% pictures grid
	\draw[gray] foreach \c in {0,...,4}{
				(p5cl cs:\c,0)  -- (p5cl cs:\c,15)
				(p5cr cs:\c,0)  -- (p5cr cs:\c,15)
					foreach \l in {0,...,15}{
						(p5cl cs:\c,\l)  -- (p5cr cs:\c,\l)}};
	% text grid
	\draw[teal] foreach \c in {0,...,4}{
					foreach \l in {0,...,14}{
						(t5cl cs:\c,\l)  -- (t5cr cs:\c,\l)}};
	% margin
	\draw[col] (p5cr cs:1,-2) node (x) {} -- (x |- N);}

% draw full grid
\newcommand\drawfullgrid{
	\draw[gray] foreach \x in {-60,...,60}{
					(page cs:\x,-45) -- (page cs:\x,45)}
				foreach \y in {-45,...,45}{
					(page cs:-60,\y) -- (page cs:60,\y)};}



% Environment for map and title page
\NewEnviron{maplayout}{

\begin{frame}[plain, noframenumbering]
\makebox[\textwidth][c]{
\begin{tikzpicture}[overlay, remember picture, inner sep=0pt]%

% main title
\node [anchor=base west, align=left, font=\LARGE, black!80]
	at (page cs:-44,23) {Modélisation de pompes à chaleur\\
						 air-air à capacité variable (PàCCV)};

% test bench icon
\filldraw [hls1!30] (p5cl cs:0,10)++(8\quanta,0) -| ++(7\quanta,5\quanta)
	-- ++(-3.5\quanta,3\quanta) -- ++(-3.5\quanta,-3\quanta) -- cycle;
\filldraw [hls1!30] (p5cr cs:0,10)++(8\quanta,0) -- ++(0,5\quanta) --
	++(-3.5\quanta,3\quanta) -- ++(-3.5\quanta,-3\quanta) |- cycle;
\draw [very thick, hls1!30]
	(p5cl cs:0,10)++(15\quanta,\quantum) coordinate (C1)
	-- +(\colv-14\quanta,0)
	(C1)++(0,2\quanta) -- +(\colv-14\quanta,0);


% control icon
\setlength\templen{0.8\baselineskip}

\begin{scope}[shift={($(p5cl cs:0,6)+(8\quanta,0)$)},
			  x=2\baselineskip, y=2\baselineskip]

\setlength\w{1.2\templen}
\setlength\e{\dimexpr \colv - 4\w \relax}

\draw [hls2!30, -{Stealth[scale=.5]}, thick]
	(0,0.8) -- ++(\w,0) ++(\w,0) -- ++(\e,0) ++(\w,0) -- ++(\w,0);
\filldraw [hls2!30] (\w,1) rectangle ++(\w,-\templen)
				 (2\w+\e,1) rectangle ++(\w,-\templen)
				 (.5\colv+.4\w,0) -- coordinate (Ain) ++(0,\templen) --
				 ++(-.8\w,-.5\templen) -- cycle;
\draw [hls2!30, -{Stealth[scale=.5]}, thick]
	(\colv-.5\w,.8) |- (Ain) ++(-.8\w,0) -| (.5\w,.8);


\end{scope}


% performance icon
\begin{scope}[shift={($(p5cl cs:0,2)+(8\quanta,0)$)},
			  x=2\baselineskip, y=2\baselineskip]

\setlength\buffer{0.48\baselineskip}
\filldraw [hls3!30]
	foreach \X in {0,1}{
	 	foreach \Y in {0,1}{
	 		foreach \x in {0,1}{
	 			foreach \y in {0,1}{
		(\X*1.2*\baselineskip, \Y*1.2*\baselineskip)++(\x\buffer,\y\buffer)
		rectangle ++(0.4\templen,0.4\templen)}}}};

\draw [hls3!30, thick] (\colv,0)++(-1,0) coordinate (O) -- ++(1,1);
\draw [semithick, hls3!30] (\colv,0) -| ++(-1,1);
\foreach \Point in {{.2,.15}, {.2,.3}, {.3,.31}, {.4,.34}, {.4,.55},
					{.55,.47}, {.6,.75}, {.7,.45}, {.75,.6}, {.8,.9},
					{.9,.97}}{
	\node [dot, scale=.7, hls3!30] at ($(O)+(\Point)$) {};}

\end{scope}


\BODY


\fill (page cs:-60,-45) rectangle (page cs:60,-48);

\end{tikzpicture}}
\end{frame}
}



% map[i] highlights section i
\newcommand{\map}[1][0]{

\pgfmathparse{
    ifthenelse(#1==1, "col", "black")}%
\colorlet{hls1}{\pgfmathresult}
\pgfmathparse{
    ifthenelse(#1==2, "col", "black")}%
\colorlet{hls2}{\pgfmathresult}
\pgfmathparse{
    ifthenelse(#1==3, "col", "black")}%
\colorlet{hls3}{\pgfmathresult}

\begin{maplayout}

\node[anchor=base east, hls1!60] at (t5cr cs:1,11) {1};
\node[anchor=base west, hls1] at (t5cl cs:2,11)
	{Collecter les données adéquates};
\node[anchor=base west, hls1!60] at (t5cl cs:2,10)
	{\footnotesize Banc d'essai et calcul des capacités};

\node[anchor=base east, hls2!60] at (t5cr cs:1,7) {2};
\node[anchor=base west, hls2] at (t5cl cs:2,7) {Modéliser le contrôle};
\node[anchor=base west, hls2!60] at (t5cl cs:2,6)
	{\footnotesize Fréquence, mode d'opération et dégivrage};

\node[anchor=base east, hls3!60] at (t5cr cs:1,3) {3};
\node[anchor=base west, hls3] at (t5cl cs:2,3)
	{Compléter les performances};
\node[anchor=base west, hls3!60] at (t5cl cs:2,2)
	{\footnotesize Corrections pour la fréquence et l'humidité};

\end{maplayout}
}




%% Plot options
\pgfkeys{/pgfplots/plot options/.style={
    separate axis lines,
    enlargelimits=false,
    axis x line*=bottom,
    axis x line shift=4\quanta,
    axis y line shift=4\quanta,
    axis y line*=left,
    scale only axis,
    ticklabel style={color=gray},
    tickwidth=\quantum,
    axis line style={gray!80!white, semithick},
    major tick style={gray!80!white, semithick},
    ticklabel style={font=\scriptsize},
    xticklabel style={yshift=\ssMheight-3\quanta},
    yticklabel style={xshift=\ssMheight-3\quanta},}}



%% Footline

% white on black footline, to set it apart from the slide
\setbeamercolor{head/foot}{fg=white, bg=black}
\setbeamertemplate{footline}{
\begin{beamercolorbox}[wd=\paperwidth,
					   ht=\flheight,
					   dp=\fldepth]{head/foot}
%	\makebox[.3333\paperwidth]{\insertsection}
%	\makebox[.3333\paperwidth]{\itshape\insertsubsection}
	\hspace*{8\quanta}\insertsection{}
	\hfill\insertframenumber{}\hspace*{8\quanta}
\end{beamercolorbox}%
}
\beamertemplatenavigationsymbolsempty  % remove navigation symbols


%% Quantities
\newcommand{\xmath}[1]{\ensuremath{#1}\xspace}  % ensure math mode

\newcommand{\Qh}[1][]{\xmath{\skew{5}\dot Q_{\text{h}#1}}}
\newcommand{\Qc}{\xmath{\skew{5}\dot Q_\text{c}}}
\newcommand{\Qcs}{\xmath{\skew{5}\dot Q_\text{cs}}}
\newcommand{\Qcl}{\xmath{\skew{5}\dot Q_\text{cl}}}
\newcommand{\Qca}{\xmath{\skew{5}\dot Q}}
\newcommand{\Pel}{\xmath{P_\text{el}}}
\newcommand{\Pcp}{\xmath{P_\text{comp}}}
\newcommand{\Pfi}{\xmath{P_\text{fi}}}
\newcommand{\fc}[1][]{\xmath{f_{c#1}}}
\newcommand{\fcn}[1][]{\xmath{\nu_{#1}}}
\newcommand{\Tr}[1][]{\xmath{T_{r#1}}}
\newcommand{\Twbr}[1][]{\xmath{T_{{\text{wb}r#1}}}}
\newcommand{\Ts}{\xmath{T_s}}
\newcommand{\To}[1][]{\xmath{T_{o#1}}}
\newcommand{\Tset}{\xmath{T_\text{set}}}
\newcommand{\dTr}{\xmath{\Delta T_r}}
\newcommand{\mr}{\xmath{\skew{5}\dot m_\text{r}}}
\newcommand{\ma}[1][]{\xmath{\skew{5}\dot m_{\text{a}#1}}}
\newcommand{\tdf}{\xmath{\tau_\text{df}}}
\newcommand{\trec}{\xmath{\tau_\text{rec}}}
\newcommand{\tss}{\xmath{\tau_\text{ss}}}
\newcommand{\theat}{\xmath{\tau_\text{h}}}
\newcommand{\Va}[1][]{\xmath{\skew{3}\dot V_{\hskip-1pt\text{a}#1}}}
\newcommand{\nph}[1]{\xmath{\varphi_\text{h#1}}}
\newcommand{\npc}[1]{\xmath{\varphi_\text{c#1}}}
\newcommand{\npcs}[1]{\xmath{\varphi_\text{cs#1}}}
\newcommand{\npcl}[1]{\xmath{\varphi_\text{cl#1}}}




% shaded paths
\makeatletter
\newif\iftikz@shading@path

\tikzset{
    % There are three circumstances in which the fading sep is needed:
    % 1. Arrows which do not update the bounding box (which is most of them).
    % 2. Line caps/joins and mitres that extend outside the natural bounding 
    %    box of the path (these are not calculated by PGF).
    % 3. Other reasons that haven't been anticipated.
    shading xsep/.store in=\tikz@pathshadingxsep,
    shading ysep/.store in=\tikz@pathshadingysep,
    shading sep/.style={shading xsep=#1, shading ysep=#1},
    shading sep=0.0cm,
}

\def\tikz@shadepath#1{% 
    % \tikz@addmode installs the `modes' (e.g., fill, draw, shade) 
    % to be applied to the path. It isn't usualy for doing more
    % changes to the path's construction.
    \iftikz@shading@path%
    \else%
        \tikz@shading@pathtrue%
        % Get the current path.
        \pgfgetpath\tikz@currentshadingpath%
        % Get the shading sep without setting any other keys.
        \begingroup%
            \pgfsys@beginscope% <- may not be necessary
            \tikzset{#1}%
            \xdef\tikz@tmp{\noexpand\def\noexpand\tikz@pathshadingxsep{\tikz@pathshadingxsep}%
                \noexpand\def\noexpand\tikz@pathshadingysep{\tikz@pathshadingysep}}%
            \pgfsys@endscope%
        \endgroup
        \tikz@tmp%
        % Get the boudning box of the current path size including the shading sep
        \pgfextract@process\pgf@shadingpath@southwest{\pgfpointadd{\pgfqpoint{\pgf@pathminx}{\pgf@pathminy}}%
            {\pgfpoint{-\tikz@pathshadingxsep}{-\tikz@pathshadingysep}}}%%
        \pgfextract@process\pgf@shadingpath@northeast{\pgfpointadd{\pgfqpoint{\pgf@pathmaxx}{\pgf@pathmaxy}}%
            {\pgfpoint{\tikz@pathshadingxsep}{\tikz@pathshadingysep}}}%
        % Clear the path
        \pgfsetpath\pgfutil@empty%                          
        % Save the current drawing mode and options.
        \let\tikz@options@saved=\tikz@options%
        \let\tikz@mode@saved=\tikz@mode%
        \let\tikz@options=\pgfutil@empty%
        \let\tikz@mode=\pgfutil@empty%
        % \tikz@options are processed later on.
        \tikz@addoption{%
            \pgfinterruptpath%
            \pgfinterruptpicture%
                \begin{tikzfadingfrompicture}[name=.]
                \pgfscope%
                    \tikzset{shade path/.style=}% Make absolutely sure shade path is not inherited.
                    \path \pgfextra{%
                        % Set the softpath. Any transformations,draw=none} in #1 will have no effect.
                        % This will *not* update the bounding box...
                        \pgfsetpath\tikz@currentshadingpath%
                        % ...so it is done manually.
                        \pgf@shadingpath@southwest
                        \expandafter\pgf@protocolsizes{\the\pgf@x}{\the\pgf@y}%
                        \pgf@shadingpath@northeast%
                        \expandafter\pgf@protocolsizes{\the\pgf@x}{\the\pgf@y}%
                        % Install the drawing modes and options.
                        \let\tikz@options=\tikz@options@saved%
                        \let\tikz@mode=\tikz@mode@saved%
                    };
                    % Now get the bounding box of the picture.
                    \xdef\pgf@shadingboundingbox@southwest{\noexpand\pgfqpoint{\the\pgf@picminx}{\the\pgf@picminy}}%
                    \xdef\pgf@shadingboundingbox@northeast{\noexpand\pgfqpoint{\the\pgf@picmaxx}{\the\pgf@picmaxy}}%
                    \endpgfscope
                \end{tikzfadingfrompicture}%
            \endpgfinterruptpicture%
            \endpgfinterruptpath%
            % Install a rectangle that covers the shaded/faded path picture.
            \pgftransformreset%
            \pgfpathrectanglecorners{\pgf@shadingboundingbox@southwest}{\pgf@shadingboundingbox@northeast}%
            %
            % Reset all modes.
            \let\tikz@path@picture=\pgfutil@empty%
            \tikz@mode@fillfalse%
            \tikz@mode@drawfalse%
            %\tikz@mode@tipsfalse%   <- To have successful compilation with pgf-tikz v3.0.1a
            \tikz@mode@doublefalse%
            \tikz@mode@clipfalse%
            \tikz@mode@boundaryfalse%
            \tikz@mode@fade@pathfalse%
            \tikz@mode@fade@scopefalse%
            % Now install shading options.
            \tikzset{#1}%
            \tikz@mode%
            % Make the fading happen.
            \def\tikz@path@fading{.}%
            \tikz@mode@fade@pathtrue%
            \tikz@fade@adjustfalse%
            % Shift the fading to the mid point of the rectangle
            \pgfpointscale{0.5}{\pgfpointadd{\pgf@shadingboundingbox@southwest}{\pgf@shadingboundingbox@northeast}}%
            \edef\tikz@fade@transform{shift={(\the\pgf@x,\the\pgf@y)}}%
            \pgfsetfading{\tikz@path@fading}{\tikz@do@fade@transform}%
            \tikz@mode@fade@pathfalse%              
        }%
    \fi%
}
\tikzset{
    shade path/.code={%
        \tikz@addmode{\tikz@shadepath{#1}}%
    }
}
\makeatother % <- To close the \makeatletter call
