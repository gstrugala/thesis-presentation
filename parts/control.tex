% MAP: PID
\begin{slide}

\begin{scope}[shift={(p5cl cs:0,13)}, node distance=5mm,
			  font=\footnotesize, inner sep=3pt]

\node[anchor=west] (Tset) at (.1,0) {$ T_\text{set} $};
\node[right=of Tset, sum] (sum1) {$ \sum $};
\coordinate[dot, right=1cm of sum1] (N1);

\node[block, anchor=north west] (ki) at ($(N1)+(.4,-1cm)$)
	{$ \dfrac{k_p}{\tau_i} $};
\node[right=1cm of ki, block] (I) {$ \dfrac{1}{s} $};

\coordinate (C1) at ($(I.east)+(.3,0)$);
\node[sum] (sum3) at (C1 |- N1) {$\sum$};
\node[block] (kp) at ($(N1)!0.5!(sum3)$) {$ k_p $};
\coordinate[right=1cm of sum3] (N2) {};

\node[block, below=25mm of kp, inner sep=5pt] (sys) {system};
\node[block, inner sep=5pt] (opp2) at (sys -| sum1) {$ -1 $};

\foreach \start/\stop in {Tset/sum1, sum1/N1, N1/kp, kp/sum3,
						  ki/I, sys/opp2, opp2/sum1}{
	\draw[-latex] (\start) -- (\stop);}
\draw[-latex] (N1) |- (ki);
\draw[-latex] (I)  -| (sum3);
\draw[-latex] (sum3) -- (N2) |- (sys);

% labels
\node[above] at (N1) {$ e $};
\node[fill=white]
	at ($(opp2)!0.5!(sum1)$) {$ \,-\Tr\phantom{-} $};
\node[fill=white] at ($(sys)!0.5!(sys -| opp2)$) {\Tr};
\node[fill=white] at ($(N2)!0.5!(N2 |- sys)$) {$ \hat\fc $};

	  
\end{scope}


\end{slide}





% operating mode
\begin{slide}[Le mode d'opération est sélectionné selon\\
			  la valeur de l'erreur et une bande morte $ 2\delta $]

%\drawhelplines

\begin{scope}[shift={(p5cl cs:2,7)}, inner sep=3pt,
			  x=\colv, y=\baselineskip]

\draw [midarrow=.83-latex, thick] (0,0) -| (.8,3);
\draw [midarrow=.83-latex, thick] (1,3)  -| (.2,0);

\draw [gray] (.2,-.75) -- (.2,-1) node [below] {$ -\delta\phantom{-} $}
						-| node [below] (d) {$ \delta $} (.8,-.75);
						
\node [right=6mm of d.base, anchor=base west] {$ e = \Tset - \Tr $};


\node [gray, inner sep=0pt, anchor=east, visible on=<1>]
	at (-\baselineskip,3) {heating};
\node [gray, inner sep=0pt, anchor=east, visible on=<1>]
	at (-\baselineskip,0) {cooling};

\draw [gray, visible on=<2>] (-\baselineskip, 0)++(\quantum,0) --
	(-\baselineskip,0) node [left] {$ -1 $} |-
	node [left] (g) {1} ++(\quantum,3);
	
\node [above=2mm of g, visible on=<2>] {$ \gamma $};

\end{scope}

\end{slide}





% MAP: operating mode
\begin{slide}

\begin{scope}[shift={(p5cl cs:0,13)}, node distance=5mm,
			  font=\footnotesize, inner sep=3pt]

\node[anchor=west] (Tset) at (.1,0) {$ T_\text{set} $};
\node[right=of Tset, sum] (sum1) {$ \sum $};
\node[right=5mm of sum1, block, fill=col!50] (mode) {$ \gamma $};
\coordinate[dot, right=5mm of mode] (N1);

\node[block, anchor=north west] (ki) at ($(N1)+(.4,-1cm)$)
	{$ \dfrac{k_p}{\tau_i} $};
\node[right=1cm of ki, block] (I) {$ \dfrac{1}{s} $};

\coordinate (C1) at ($(I.east)+(.3,0)$);
\node[sum] (sum3) at (C1 |- N1) {$\sum$};
\node[block] (kp) at ($(N1)!0.5!(sum3)$) {$ k_p $};
\coordinate[right=1cm of sum3] (N2) {};

\node[block, below=25mm of kp, inner sep=5pt] (sys) {system};
\node[block, inner sep=5pt] (opp2) at (sys -| sum1) {$ -1 $};

\foreach \start/\stop in {Tset/sum1, sum1/mode, mode/N1, N1/kp, kp/sum3,
						  ki/I, sys/opp2, opp2/sum1}{
	\draw[-latex] (\start) -- (\stop);}
\draw[-latex] (N1) |- (ki);
\draw[-latex] (I)  -| (sum3);
\draw[-latex] (sum3) -- (N2) |- (sys);

% labels
\node[above, col] at (N1) {$ \gamma e $};
\node[fill=white]
	at ($(opp2)!0.5!(sum1)$) {$ \,-\Tr\phantom{-} $};
\node[fill=white] at ($(sys)!0.5!(sys -| opp2)$) {\Tr};
\node[fill=white] at ($(N2)!0.5!(N2 |- sys)$) {$ \hat\fc $};

\end{scope}


\end{slide}





% frequency limits
\begin{slide}[La fréquence est confinée dans un intervalle\only<2>{,\\}
			  \only<2>{et quantifiée sur des valeurs discrètes}]

\setlength\templen{\dimexpr\colvsep/2-5mm\relax}
\setlength\buffer{%
	\dimexpr0.5\colvsep-0.125\baselineskip - 0.25\templen\relax}

% saturation
\begin{scope}[visible on=<1>]

% saturation block
\node [block, minimum width=1cm, minimum height=1cm] (sat)
	at ($(p5cl cs:2,9)+(\colv/2,0)$) {};
\draw (sat.south west)++(.2,.2) -- ++(.2,0) -- ($(sat.north east)-(.4,.2)$)
	  -- ++(.2,0);
\coordinate [right=1cm of sat.east] (C4);

% signals
\begin{scope}[shift={(p5cl cs:0,9)}, x=\buffer]

% unsaturated signal
\draw[col, very thick, shorten >=-.2pt] plot[samples=100, domain=0:2]
	(\x, {0.5*sin(600*(\x-.02)) * exp(-2*\x)}) coordinate (C1);

\begin{scope}[xshift=2\buffer, yshift=-5mm]
	\draw[col, very thick, shorten <=-.6pt] plot[samples=100, domain=0:2]
		(\x, {0.5*cos(600*(\x)) * exp(-2*\x)}) coordinate (C2);
\end{scope}


\node [becomes, rotate=-90] (B1) at ($(sat.west)-(\templen,0)$) {};
\node [becomes, rotate=-90] (B2) at ($(sat.east)+(\templen,0)$) {};


% saturated signal
\begin{scope}[xshift=\bigcol+\colvsep+2\quanta+\templen]

\draw [gray!50, very thick] plot[samples=50, domain=.05:.28]
	(\x, {0.5*sin(600*(\x-.02)) * exp(-2*\x)});
	
\draw [very thick, col, shorten >=-.5pt] plot[samples=50, domain=0:.06]
	(\x, {0.5*sin(600*(\x-.02)) * exp(-2*\x)});

\draw [very thick, col, shorten <=-.2pt, shorten >=-.2pt]
	(.05,.1804) -- (.267,.1804) coordinate (fmax);
	
\draw [very thick, col, shorten <=-.4pt, shorten >=-.1pt]
	plot[samples=100, domain=.258:2]
		(\x, {0.5*sin(600*(\x-.02)) * exp(-2*\x)});
	
\begin{scope}[xshift=2\buffer, yshift=-5mm]
	\draw [gray!50, very thick] plot[samples=50, domain=.16:.42]
		(\x, {0.5*cos(600*(\x)) * exp(-2*\x)});
	
	\draw [very thick, col, shorten <=-.6pt, shorten >=-.4pt]
		plot[samples=50, domain=0:.18]
			(\x, {0.5*cos(600*(\x)) * exp(-2*\x)});
	
	\draw [very thick, col, shorten <=-.2pt, shorten >=-.2pt]
		(.17,-.10777) -- (.41,-.10777) coordinate (fmin);
	
	\draw [very thick, col, shorten <=-.4pt]
		plot[samples=100, domain=.402:2]
			(\x, {0.5*cos(600*(\x)) * exp(-2*\x)}) coordinate (C6);

\end{scope}

\coordinate (O);
\coordinate (C7) at (1.7,0);
\coordinate (fmaxl) at (C7 |- fmax);
\coordinate (fminl) at (C7 |- fmin);

\end{scope}


\begin{scope}[on background layer]
	\filldraw[gray!10, visible on=<1>] (fmax -| O)++(0,.4pt) rectangle
		($(fmin -| C6)-(0,.4pt)$);
\end{scope}

\draw [latex-] (fmaxl) -- ++(0,\baselineskip);
\draw [latex-] (fminl) -- ++(0,-\baselineskip)
	node [below, inner sep=3pt] {\footnotesize $ \Delta f $};

\end{scope}

\end{scope}


\begin{scope}[visible on=<2>]

% quantization block
\node[block, minimum width=1cm, minimum height=1cm] (qt)
	at ($(p5cl cs:2,9)+(\colv/2,0)$) {};
\draw[col, thick] (qt.south west)++(.2,.2) cos ++(.03,.1) sin ++(.12,.45)
					cos ++(.15,-.15) sin ++(.3,-.2);
\draw (qt.south west)++(.2,.2) |- ++(.1,.2) |- ++(.1,.4) |- ++(.3,-.2)
	  |- ++(.1,-.3);
\coordinate[right=1cm of qt.east] (C4);

% signals
\begin{scope}[shift={(p5cl cs:0,9)}, x=\buffer]

% saturated signal
\draw [gray!50, very thick] plot[samples=50, domain=.05:.28]
	(\x, {0.5*sin(600*(\x-.02)) * exp(-2*\x)});
	
\draw [very thick, col, shorten >=-.5pt] plot[samples=50, domain=0:.06]
	(\x, {0.5*sin(600*(\x-.02)) * exp(-2*\x)});

\draw [very thick, col, shorten <=-.2pt, shorten >=-.2pt]
	(.05,.1804) -- (.267,.1804) coordinate (fmax);
	
\draw [very thick, col, shorten <=-.4pt, shorten >=-.1pt]
	plot[samples=100, domain=.258:2]
		(\x, {0.5*sin(600*(\x-.02)) * exp(-2*\x)});
	
\begin{scope}[xshift=2\buffer, yshift=-5mm]
	\draw [gray!50, very thick] plot[samples=50, domain=.16:.42]
		(\x, {0.5*cos(600*(\x)) * exp(-2*\x)});
	
	\draw [very thick, col, shorten <=-.6pt, shorten >=-.4pt]
		plot[samples=50, domain=0:.18]
			(\x, {0.5*cos(600*(\x)) * exp(-2*\x)});
	
	\draw [very thick, col, shorten <=-.2pt, shorten >=-.2pt]
		(.17,-.10777) -- (.41,-.10777) coordinate (fmin);
	
	\draw [very thick, col, shorten <=-.4pt]
		plot[samples=100, domain=.402:2]
			(\x, {0.5*cos(600*(\x)) * exp(-2*\x)}) coordinate (C6);

\end{scope}


\node [becomes, rotate=-90] (B1) at ($(qt.west)-(\templen,0)$) {};
\node [becomes, rotate=-90] (B2) at ($(qt.east)+(\templen,0)$) {};


% quantized signal
\begin{scope}[xshift=\bigcol+\colvsep+2\quanta+\templen]

\draw [very thick, gray!50, shorten >=-.3pt]
	plot[samples=50, domain=0:.06]
		(\x, {0.5*sin(600*(\x-.02)) * exp(-2*\x)});

\draw [very thick, gray!50] (.05,.1804) -- (.267,.1804);
	
\draw[very thick, gray!50, shorten <=-.3pt]
	plot[samples=100, domain=.258:2]
		(\x, {0.5*sin(600*(\x-.02)) * exp(-2*\x)}) coordinate (C1);
	
\begin{scope}[xshift=2\buffer, yshift=-5mm]
	
	\draw [very thick, gray!50, shorten <=-.5pt, shorten >=-.3pt]
		plot[samples=50, domain=0:.18]
			(\x, {0.5*cos(600*(\x)) * exp(-2*\x)});
	
	\draw [very thick, gray!50] (.17,-.10777) -- (.41,-.10777);
	
	\draw [very thick, gray!50, shorten <=-.3pt]
		plot[samples=100, domain=.402:2]
			(\x, {0.5*cos(600*(\x)) * exp(-2*\x)}) coordinate (C2);
\end{scope}

\draw [col, very thick] (0,-.1) |- (.2,.1804) |- (.4, .05) |- (.6, -.2)
	|- (.8,.1) |- (1,0) |- (1.2,-.1) |- (1.4,.05) |- (2,0) |- (2.2,-.3)
	|- (2.4,-.6078) |- (2.8,-.4) |- (3,-.55) |- (3.2,-.45) |- (4,-.5);

\end{scope}

\end{scope}

\end{scope}


\end{slide}





% MAP: frequency limits & quantization
\begin{slide}

\begin{scope}[shift={(p5cl cs:0,13)}, node distance=5mm,
			  font=\footnotesize, inner sep=3pt]

\node[anchor=west] (Tset) at (.1,0) {$ T_\text{set} $};
\node[right=of Tset, sum] (sum1) {$ \sum $};
\node[right=5mm of sum1, block] (mode) {$ \gamma $};
\coordinate[dot, right=5mm of mode] (N1);

\node[block, anchor=north west] (ki) at ($(N1)+(.4,-1cm)$)
	{$ \dfrac{k_p}{\tau_i} $};
\node[right=of ki, sum, fill=col!50, visible on=<2>] (sum2) {$ \sum $};
\node[right=of sum2, block, hlbg=fill on <2>] (I) {$ \dfrac{1}{s} $};

\node[block] (kp) at (N1 -| sum2) {$ k_p $};
\coordinate (C1) at ($(I.east)+(.3,0)$);
\node[sum] (sum3) at (C1 |- kp) {$\sum$};
\coordinate[dot, right=10mm of sum3] (N2) {};

% saturation block
\node[block, right=11mm of N2, minimum width=1cm, minimum height=1cm,
	  hlbg=fill on <1>] (sat) {};
\draw (sat.south west)++(.2,.2) -- ++(.2,0) -- ($(sat.north east)-(.4,.2)$)
	  -- ++(.2,0);

% quantizer
\node[block, below=of sat, minimum width=1cm, minimum height=1cm,
	  hlbg=fill on <1>] (qt) {};
\draw[col, thick] (qt.south west)++(.2,.2) cos ++(.03,.1) sin ++(.12,.45)
					cos ++(.15,-.15) sin ++(.3,-.2);
\draw (qt.south west)++(.2,.2) |- ++(.1,.2) |- ++(.1,.4) |- ++(.3,-.2)
	  |- ++(.1,-.3);

\node[block, inner sep=5pt, fill=col!50, visible on=<2>] (opp)
	at (N2 |- I) {$ -1 $};
\node[sum, below=1cm of opp, fill=col!50, visible on=<2>] (sum4)
	{$ \sum $};
\coordinate[dot] (N3) at (sum4 -| qt) {};
\node[block, fill=col!50, visible on=<2>] (kt)
	at (sum4 -| I) {$ \dfrac{1}{\tau_t} $};

\node[block, below=1cm of kt, inner sep=5pt] (sys) {system};
\node[block, inner sep=5pt] (opp2) at (kt -| sum1) {$ -1 $};

\foreach \start/\stop in {Tset/sum1, sum1/mode, mode/N1, N1/kp, kp/sum3,
						  sum3/N2, N2/sat, sat/qt, qt/N3, opp2/sum1}{
	\draw[-latex] (\start) -- (\stop);}
\foreach \start/\stop in {N3/sum4, N2/opp, opp/sum4, sum4/kt,
						  ki/sum2, sum2/I}{
	\draw[-latex, draw on=<2>] (\start) -- (\stop);}
\draw[-latex, visible on=<1>] (ki) -- (I);
\draw[-latex] (N1) |- (ki);
\draw[-latex] (I)  -| (sum3);
\draw[-latex, draw on=<2>] (kt) -| (sum2);
\draw[-latex] (N3) |- (sys);
\draw[-latex] (sys) -| (opp2);

% labels
\node[above] at (N1) {$ \gamma e $};
\node[fill=white] at ($(opp2)!0.5!(sum1)$) {$ \,-\Tr\phantom{-} $};
\node[fill=white] at ($(sys.west)!0.5!(sys -| opp2)$) {\Tr};
\node[above] at (N2) {$ \hat\fc $};
\node[right] at (N3) {\fc};


\end{scope}

\end{slide}





% defrost cycles
\begin{slide}[Un cycle de dégivrage est divisé en 3 phases
			  \only<2->{:}\\
			  \only<2->{\textcolor<3->{gray}{le dégivrage}}%
			  \only<3->{\textcolor<3->{gray}{,}}
			  \only<3->{\textcolor<4->{gray}{la récupération}}%
			  \only<4->{\textcolor<4->{gray}{,}}
			  \only<4->{le régime permanent}]

\begin{scope}[shift={(p5cl cs:2,7)}, x=\colv, y=\baselineskip]

\draw[very thick, shorten >=-.33pt, hlg=draw on <1-2>]
	(-\baselineskip,0) -- (0,0);
\draw[very thick, shorten <=-.33pt, hlg=draw on <{1,3}>]
	plot[domain=0:1] (\x, {6*\x-3*\x^2});
\draw[very thick, hlg=draw on <{1,4}>]
	(\colv,3) -- ++(\colvsep,0);

% axes
\begin{scope}[shift={(-\baselineskip,4\baselineskip)},
			  x=\baselineskip, inner sep=3pt, font=\footnotesize]

\draw[gray, semithick] (0,0) -- node[above]{\tdf} (1,0) --
	node[above]{\trec} ++(\colv,0) -- node[above]{\tss} ++(\colvsep,0);
\draw[gray, semithick] % ticks
	(0,.3pt) -- +(0,-\quantum) ++(1,0) -- +(0,-\quantum)
	++(\colv,0) -- +(0,-\quantum) ++(\colvsep,0) -- +(0,-\quantum);
\draw[gray, semithick] (2\colvsep+3\quanta,-1) -- ++(.25,0)
	node[right] {\Qh[\text{ss}]} |- node[right] {0} ++(-.25,-3);

\end{scope}

% defrost parts
\begin{scope}[shift={(-\baselineskip,-\baselineskip)},
			  x=\baselineskip, inner sep=3pt, font=\footnotesize]

\draw[semithick, visible on=<2>] (0,.25) -- (0,0) -|
	node[below, pos=.25] {$ \Qh = 0 $} (1,.25);
\draw[semithick, visible on=<3>] (1,.25) -- (1,0) -|
	node[below, pos=.25] (rec) {$ \Qh = \epsilon\Qh[\text{ss}] $}
	++(\colv,.25);
\draw[semithick, visible on=<4>] (\colvsep,.25) -- ++(0,-.25) -|
	node[below, pos=.25] {$ \Qh = \Qh[\text{ss}] $} ++(\colvsep,.25);

\draw[visible on=<3>] (rec.293) -- ++(0,-\baselineskip)
	node[below, fill=gray!20, yshift=-3pt]
{$ \epsilon = 2\dfrac{t}{\trec} - \left( \dfrac{t}{\trec} \right)^2 $};
\end{scope}

\end{scope}

\end{slide}





% defrost parameters
\begin{slide}[Les paramètres \trec et \theat\\
			  dépendent de la température ambiante]

\begin{scope}[shift={(p5cl cs:2,7)}, x=\colv, y=\baselineskip]

\draw[col, very thick, shorten >=-.33pt]
	(-\baselineskip,0) -- (0,0);
\draw[col, very thick, shorten <=-.33pt]
	plot[domain=0:1] (\x, {6*\x-3*\x^2}) -- ++(\colvsep,0);

% axes
\begin{scope}[shift={(-\baselineskip,4\baselineskip)},
			  x=\baselineskip, inner sep=3pt, font=\footnotesize]

\draw[gray, semithick] (0,0) -- node[above]{\tdf} (1,0) --
	node[above] (trec) {\trec} ++(\colv,0) --
	node[above] {\tss} ++(\colvsep,0);
\draw[gray, semithick] % ticks
	(0,.3pt) -- +(0,-\quantum) ++(1,0) -- +(0,-\quantum)
	++(\colv,0) -- +(0,-\quantum) ++(\colvsep,0) -- +(0,-\quantum);
\draw[gray, semithick] (2\colvsep+3\quanta,-1) -- ++(.25,0)
	node[right] {\Qh[\text{ss}]} |- node[right] {0} ++(-.25,-3);

\end{scope}


\begin{scope}[shift={(-\baselineskip,-\baselineskip)},
			  x=\baselineskip, inner sep=3pt, font=\footnotesize]

\draw[gray, semithick] (0,0) -- node[below] (tdf) {\tdf} (1,0) --
	node[below] (theat) {\theat} ++(\colv+\colvsep,0);
\draw[gray, semithick] % ticks
	(0,-.3pt) -- +(0,\quantum) ++(1,0) -- +(0,\quantum)
	++(\colv+\colvsep,0) -- +(0,\quantum);

\draw (tdf.south) -- ++(0,-\baselineskip)
	node[below] {\SI{4.64}{\minute}};
\draw (theat.south) -- ++(0,-\baselineskip)
	node[below] {fonction de \To};
\draw (trec.north) -- ++(0,\baselineskip)
	node[above] {fonction de \To};

\end{scope}

\end{scope}

\end{slide}





% defrost regressions
\begin{slide}[Le temps de récupération est déterminé\\
			  avec une régression de la température ambiante]

\begin{axis}[plot options, clip=false,
			 clip mode=individual,
			 at={(p5cl cs:2,3)},
			 width=\bigcol+2\baselineskip, height=10\baselineskip,
			 xmin=-19, xmax=7,
			 xtick={-19, 7},
			 xticklabels={\num{-19}\phantom{$-$}, \SI{7}{\celsius}},
			 ymin=6, ymax=34,
			 ytick={6, 34},
			 yticklabels={6, \SI{34}{\minute}},]

\addplot[only marks, mark size=.7pt]
	table[x=Toa, y=t_rec] {data/recovery-time.tsv};
\addplot[col, very thick, domain=-19:5.15] {10.798 - 0.93146*x};
%\coordinate (mid) at (axis cs:-5.5,40.16);

\end{axis}


\node[anchor=mid west, font=\footnotesize]
	at (p5cl cs:1,13) {\trec};
\node[anchor=base east, font=\footnotesize]
	at (t5cr cs:4,1) {\To};

\node[anchor=base west, font=\footnotesize] (reg)
	at (t5cl cs:3,11) {$ 10.798 - 0.9315\,\frac{\To}{\si{\celsius}} $};
\draw[col] (reg.south)++(0,-6pt) -- ++(0,-4\baselineskip);


\end{slide}





% defrost regressions
\begin{slide}[Le temps d'opération en chauffage est déterminé\\
			  avec une régression de la température ambiante]

\begin{axis}[plot options, clip=false,
			 clip mode=individual,
			 at={(p5cl cs:2,3)},
			 width=\bigcol+2\baselineskip, height=10\baselineskip,
			 xmin=-18, xmax=7,
			 xtick={-18, 7},
			 xticklabels={\num{-18}\phantom{$-$}, \SI{7}{\celsius}},
			 ymin=0, ymax=250,
			 ytick={0, 37.5, 90, 250},
			 yticklabels={0, 37.5, 90, \SI{250}{\minute}}]

\addplot[only marks, mark size=.7pt, color=gray!40, visible on=<1>]
	table[x=Toa, y=t_h] {data/heating-time-rejected.tsv};
\addplot[only marks, mark size=.7pt]
	table[x=Toa, y=t_h] {data/heating-time-kept.tsv};
\addplot[col, very thick, domain=-18:7, visible on=<2>]
	{37.3874 + 16.6383*exp(0.2352*(x - 2.1179))};
\coordinate (mid) at (axis cs:-5.5,40.16);

\node[gray!40, left, inner sep=3pt, font=\scriptsize, visible on=<1>]
	at (axis cs:2.9325,249) {rejected};
\node[right, inner sep=3pt, font=\scriptsize, visible on=<1>]
	at (axis cs:3.8619,34) {kept};

\end{axis}


\node[anchor=mid west, font=\footnotesize]
	at (p5cl cs:1,13) {\theat};
\node[anchor=base east, font=\footnotesize]
	at (t5cr cs:4,1) {\To};

\node[anchor=base west, font=\footnotesize, visible on=<2>] (reg)
	at (p5cl cs:2,10) {$ 37.387 + 16.638\,\exp\left[0.235\left(
					   \frac{\To}{\si{\celsius}} - 2.118\right)\right] $};
\draw[col, visible on=<2>]
	(reg.south -| mid)++(0,-10pt) -- ($(mid)+(0,10pt)$);


\end{slide}





% MAP: whole controller
\begin{slide}

\begin{scope}[shift={(p5cl cs:0,13)}, node distance=5mm,
			  font=\footnotesize, inner sep=3pt]

\node[anchor=west] (Tset) at (.1,0) {$ T_\text{set} $};
\node[right=of Tset, sum] (sum1) {$ \sum $};
\node[right=5mm of sum1, block] (mode) {$ \gamma $};
\coordinate[dot, right=5mm of mode] (N1);

\node[block, anchor=north west] (ki) at ($(N1)+(.4,-1cm)$)
	{$ \dfrac{k_p}{\tau_i} $};
\node[right=of ki, sum] (sum2) {$ \sum $};
\node[right=of sum2, block] (I) {$ \dfrac{1}{s} $};

\node[block] (kp) at (N1 -| sum2) {$ k_p $};
\coordinate (C1) at ($(I.east)+(.3,0)$);
\node[sum] (sum3) at (C1 |- kp) {$\sum$};
\coordinate[dot, right=10mm of sum3] (N2) {};

% saturation block
\node[block, right=11mm of N2, minimum width=1cm, minimum height=1cm]
	(sat) {};
\draw (sat.south west)++(.2,.2) -- ++(.2,0) -- ($(sat.north east)-(.4,.2)$)
	  -- ++(.2,0);

% quantizer
\node[block, below=of sat, minimum width=1cm, minimum height=1cm] (qt) {};
\draw[col, thick] (qt.south west)++(.2,.2) cos ++(.03,.1) sin ++(.12,.45)
					cos ++(.15,-.15) sin ++(.3,-.2);
\draw (qt.south west)++(.2,.2) |- ++(.1,.2) |- ++(.1,.4) |- ++(.3,-.2)
	  |- ++(.1,-.3);

\node[block, inner sep=5pt] (opp) at (N2 |- I) {$ -1 $};
\node[sum, below=1cm of opp] (sum4) {$ \sum $};
\coordinate[dot] (N3) at (sum4 -| qt) {};
\node[block] (kt) at (sum4 -| I) {$ \dfrac{1}{\tau_t} $};

\node[block, below=1cm of kt, inner sep=5pt] (sys) {system};
\node[block, inner sep=5pt] (opp2) at (kt -| sum1) {$ -1 $};

\foreach \start/\stop in {Tset/sum1, sum1/mode, mode/N1, N1/kp, kp/sum3,
						  sum3/N2, N2/sat, sat/qt, qt/N3, N3/sum4,
						  N2/opp, opp/sum4, sum4/kt,
						  ki/sum2, sum2/I, opp2/sum1}{
	\draw[-latex] (\start) -- (\stop);}
\draw[-latex] (N1) |- (ki);
\draw[-latex] (I)  -| (sum3);
\draw[-latex] (kt) -| (sum2);
\draw[-latex] (N3) |- (sys);
\draw[-latex] (sys) -| (opp2);

% labels
\node[above] at (N1) {$ \gamma e $};
\node[fill=gray!10]
	at ($(opp2)!0.5!(sum1)$) {$ \,-\Tr\phantom{-} $};
\node[fill=white] at ($(sys.west)!0.5!(sys -| opp2)$) {\Tr};
\node[above] at (N2) {$ \hat\fc $};
\node[right] at (N3) {\fc};

% components
\begin{scope}[on background layer]

\node[fill=gray, opacity=0.1, rounded corners,
	  fit=(Tset) (sum1) (sum3) (ki) (I)] (PI) {};
\node[fill=gray, opacity=0.1, rounded corners,
	  fit=(opp) (sum4) (kt) (sum2) (I)] (anti-windup) {};
	  
\end{scope}

\node[gray, anchor=south west] at (PI.north west) {bloc PI};
\node[gray, anchor=south east, align=left] at (anti-windup.south west)
	{anti-\\windup};

\end{scope}

\end{slide}
